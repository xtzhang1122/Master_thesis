% For copyright and license information, see uiucthesis2021.dtx and derivatives.
\documentclass[draft]{uiucthesis2021}
\usepackage[utf8]{inputenc}
\usepackage[english]{babel}
\usepackage{csquotes}
\usepackage{microtype}
\usepackage{amsmath,amsthm,amssymb}
\usepackage[bookmarksdepth=3,linktoc=all,colorlinks=true,urlcolor=blue,linkcolor=blue,citecolor=blue]{hyperref}
\usepackage[style=ieee]{biblatex}

% \usepackage{ruledchapters}  % example of compliant heading format, uncomment to use

\usepackage{lipsum}  % just for placeholder code

% uncomment the below to show a grid on all pages
% \usepackage[grid, gridunit=in, gridcolor=blue!40, subgridcolor=blue!20]{eso-pic}

\addbibresource{./references.bib}

\newcounter{counterforappendices}

\begin{document}

\title{Domain-Aware Entropy-Selective Knowledge
Distillation}
\author{Xitong (Jacqueline) Zhang}
\department{Bioinformatics}
\concentration{Information Science}
\msthesis
\degreeyear{2025}
\committee{
    Professor Prof Jingrui He\\
    Assistant Professor Prof Jiaqi Ma}
\maketitle

\frontmatter

\begin{abstract}
This is a comprehensive study of caffeine consumption by graduate
students at the University of Illinois who are in the very final
stages of completing their doctoral degrees. A study group of six
hundred doctoral students\ldots.
\end{abstract}

\begin{dedication}
To Father and Mother.
\end{dedication}

\begin{acknowledgments}
This project would not have been possible without the support of
many people. Many thanks to my adviser, Lawrence T. Strongarm, who
read my numerous revisions and helped make some sense of the
confusion. Also thanks to my committee members, Reginald Bottoms,
Karin Vegas, and Cindy Willy, who offered guidance and support.
Thanks to the University of Illinois Graduate College for awarding
me a Dissertation Completion Fellowship, providing me with the
financial means to complete this project. And finally, thanks to
my husband, parents, and numerous friends who endured this long
process with me, always offering support and love.
\end{acknowledgments}

{
    \hypersetup{linkcolor=black}  % disable link coloring locally
    \tableofcontents
    % the Graduate College doesn't recommend including lot or lof
    % \listoftables
    % \listoffigures
}

\chapter{List of Abbreviations}

\begin{abbrevlist}
\item[CA] Caffeine Addict.
\item[CD] Coffee Drinker.
\end{abbrevlist}

\chapter{List of Symbols}

\begin{symbollist}[0.7in]
\item[$\tau$] Time taken to drink one cup of coffee.
\item[$\mu$g] Micrograms (of caffeine, generally).
\end{symbollist}

\mainmatter

\chapter{Introduction}
\section{ To be fixed: Domain Adaptation during Knowledge Distillation in Large Language Models}

The advancement of Large Language Models (LLMs) has significantly accelerated the progress of Artificial Intelligence (AI), extending the reach of Natural Language Processing (NLP) across numerous domains. 
These include language understanding \cite{brown2020language}, programming and code generation \cite{chen2021evaluating}, recommendation systems \cite{sun2023recommendation}, information retrieval \cite{ma2023instructretrieval}, mobile-device interaction and voice assistants \cite{rao2023speak}, scientific discovery \cite{nori2023capabilities}, medical question answering \cite{singhal2022large}, and legal reasoning \cite{katz2023gpt}. 
The release of powerful commercial LLMs such as ChatGPT \cite{openai2023gpt4}, Bard \cite{thoppilan2022lamda}, and Claude \cite{anthropic2023claude}, along with the open-source LLaMA models \cite{touvron2023llama, touvron2023llama2}, has spurred rapid growth in both academic research and real-world applications. 
However, the sheer scale of these models—often involving billions of parameters—introduces computational and financial burdens, limiting their accessibility in resource-constrained settings. 
While these LLMs are powerful, their size and generality make it challenging to fully harness their capabilities, and this gap in utilization can hinder their effectiveness in specialized applications.

Knowledge Distillation (KD) \cite{hinton} has emerged as a widely adopted paradigm for model compression and the acceleration of transfer learning in deep neural networks \cite{kd-survey}. 
Within the context of LLMs, KD effectively fulfills the objective of transferring the representational and reasoning capabilities of large-scale teachers to smaller, computationally efficient student models suitable for deployment \cite{xu2024survey}. 
Mathematically, KD aims to minimize the divergence (usually Kullback–Leibler divergence) between the soft output distributions of a large teacher model and a smaller student model. 
Distilling into a 7B–13B model can reduce inference cost by 10×, making high-quality language generation feasible on resource-constrained hardware. 
However, standard KD methods are both computationally intensive and vulnerable to generalization failure \cite{slmsurvey}. 
Two key challenges underlie these limitations: 
First, distribution mismatch \cite{gkd,distillm,distillm2} arises when the student fails to match the teacher’s predictive distribution over all training examples. 
Since KD typically applies uniform supervision across the dataset, it can waste teacher compute on examples that are already easy for the student—offering little training signal while duplicating the teacher's token budget.
Second, domain gap \cite{adaptanddistill} between the teacher’s pretraining distribution and the downstream deployment domain can cause up to 5–10 percentage point drops in performance unless both models undergo heavy continual pretraining \cite{adaptanddistill}. 
Such adaptation significantly increases cost, negating the benefits of distillation.

One of the fundamental challenges in KD lies in determining which training examples are most informative for the student model, thereby raising the question of how to effectively select and rank data samples for distillation. 
Although multiple approaches have demonstrated that more challenging examples tend to yield greater learning benefits \cite{eakd,dakd}, this assumption does not always hold in domain-specialized settings. 
When the student model is intended to operate within a domain for which the teacher model possesses limited knowledge, the entropy signals across samples become relatively uniform. 
Consequently, the data selection and ranking mechanism loses its discriminative power, rendering the overall distillation process less effective.

\textbf{Add a transition sentence}


\chapter{Background}
\section{Related Work}

We organize prior work into three broad directions: (i) reshaping the knowledge distillation (KD) objective, (ii) incorporating difficulty- or uncertainty-awareness into supervision, and (iii) adapting models to new domains through continued pretraining or vocabulary changes. We move from objective-level modifications, to approaches that explicitly reason about instance difficulty, and finally to domain-oriented strategies. Each thread provides context for how our proposed DA-ESKD departs from existing practices.

\paragraph{Objective reshaping.}
A first line of work modifies the KD loss while still supervising broadly across data. 
\textbf{GKD} \cite{gkd} adopts an on-policy strategy: the student trains on its own generations and receives token-level feedback via a generalized JSD objective, directly reducing the train–inference mismatch. 
\textbf{DistiLLM} \cite{distillm} and its contrastive extension \textbf{DistiLLM2} \cite{ko2025distillm} instead refine off-policy supervision, using skewed KL divergence and contrastive penalties to sharpen the student distribution on teacher-dispreferred tokens. 
\textbf{Logit Standardization (LS)} \cite{ls} tackles instability from mismatched logit scales by applying Z-score normalization before temperature scaling, preserving geometric structure prior to softmax. 
These approaches improve optimization but typically still require teacher queries across nearly the entire training corpus, so compute cost grows with dataset size.

\paragraph{Difficulty-aware supervision.}
A second set of methods introduces difficulty signals. 
Some approaches maintain full-corpus supervision but reweight examples: \textbf{EA-KD} \cite{eakd} and \textbf{RW-KD} \cite{rwkd} emphasize uncertain instances, while \textbf{ER-KD} \cite{erkd} directly scales per-sample loss by the teacher’s predictive entropy, steering the student toward high-entropy examples. 
Other approaches move beyond reweighting to actual data pruning. 
\textbf{DA-KD} \cite{dakd}, for instance, constructs a smaller but harder training set via a Distillation Difficulty Score (DDS) defined by teacher–student loss ratios, prunes easy samples, and reinjects a fraction of them for diversity. To stabilize training on these challenging subsets, it introduces a Bidirectional Discrepancy Loss (BDL) that bounds gradients and balances teacher–student divergence. Such difficulty-based methods reduce redundancy and improve efficiency, but still require extensive teacher scoring to estimate difficulty in the first place.

\paragraph{Domain adaptation.}
Orthogonal to objective and difficulty-based strategies, another large body of work addresses domain shift. 
\textbf{Adapt-and-Distill} \cite{adaptanddistill} demonstrates that continual pretraining in-domain before KD improves accuracy, though at nearly double the compute cost. 
\textbf{AdaLM} expands the tokenizer to mitigate subword drift, reducing out-of-vocabulary issues but lengthening sequences. Continued pretraining more broadly has shown consistent gains: 
\textbf{Reuse, Don’t Retrain} \cite{reuse2024} optimizes schedules for large models with $\sim$9\% improvements; 
\textbf{Gururangan et al.} \cite{gururangan2020don} report robust cross-domain benefits across multiple domains; 
\textbf{PCP} \cite{pcp2023} leverages prompt templates during CPT; and \textbf{Domain-Adaptive CPT for small LMs} \cite{domainadaptive2025} develops more efficient CPT for low-resource settings. 
\textbf{VE-KD} \cite{vekd2024} goes further by coupling vocabulary expansion with KD, surpassing Adapt-and-Distill on biomedical tasks while reducing training time. 
Beyond these general-purpose methods, several works adapt KD pipelines to specialized domains—few-shot classification \cite{adasent2023}, protein multitask regression \cite{selfprot2024}, or non-English corpora \cite{germanprocess2023}. For example, augmenting KD with a $k$-NN pipeline on German industrial IR improves accuracy while cutting GPU use by $\sim$4$\times$. 
These domain-oriented strategies consistently improve transfer, but often at the expense of higher pretraining or tokenization costs.

\paragraph{To be fixed: Our approach in context.}
\textbf{DA-ESKD} combines the strengths of selective supervision and lightweight adaptation. Rather than reweighting after full teacher computation (EA-KD/RW-KD/ER-KD) or broadly scoring before pruning (DA-KD), it gates teacher queries by the student’s own entropy, only consulting the teacher on uncertain cases. This avoids unnecessary teacher calls on easy examples. In parallel, a single-epoch masked-LM warm-up on in-domain text provides most of the domain benefit of continued pretraining without its full cost. Together, these design choices yield a favorable quality--cost trade-off by unifying compute-aware selective querying with minimal domain adaptation.


\section{Proposed Approach}
Our goal is to efficiently distill a large language model into a smaller student that performs well under domain shift while minimizing computational cost. We begin by adapting both teacher and student models to the target domain using a brief round of pretraining on a corpus of explanation-rich texts. This step ensures that both models share a more relevant vocabulary and representation space, helping to mitigate domain drift without requiring expensive continual pretraining.

Once adapted, the student estimates its own uncertainty over the training set and selectively queries the teacher only on the most difficult examples—those it is least confident about. Distillation begins with a small subset of high-uncertainty samples, and this subset is gradually expanded over time. This exploration-style schedule allows the student to learn from the most informative supervision signals while avoiding redundant teacher calls on easy examples.

The learning objective encourages the student to align closely with the teacher on the selected samples, while continuing to improve its own predictions across the broader dataset. In contrast to traditional KD pipelines, which require full teacher passes over all examples, our method concentrates effort where it matters most—achieving greater efficiency without compromising accuracy.

To validate the method, we compare it against three representative baselines: standard fine-tuning without KD, full-corpus distillation after domain adaptation, and entropy-based KD without adaptation. Models are evaluated on both open-domain and reasoning-intensive benchmarks. In addition to accuracy metrics, we also report total training cost and teacher-query volume, allowing us to assess the efficiency and practicality of the proposed pipeline.



\chapter{Conclusions}

We conclude that graduate students like coffee.

% per Graduate College preference, place the \appendix and the appendices content before the
% bibliography (here) only if the appendices contain references.

\backmatter

\printbibliography[heading=bibintoc,title={References}]

% the below lines are only needed if bibliography precedes appendices
% uses https://tex.stackexchange.com/a/440212 to continue page numbering
% \clearpage
% \setcounter{counterforappendices}{\value{page}}
% \mainmatter



% \setcounter{page}{\value{counterforappendices}}

% \appendix

% \chapter{An appendix}


% \input{Appendix.tex}

\end{document}
